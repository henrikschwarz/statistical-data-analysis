\documentclass{article}

\author{Henrik Schwarz}

\usepackage{amssymb,amsmath}
\usepackage{graphicx}

\graphicspath{ {./media/} }

\title{Statistical Data Analysis \\ \small{EKA: T510028102}}

\date{}

\newcommand{\paraheading}[1]{\paragraph{#1} \mbox{}} % create a paragraph and then a mbox so breaking line is possible

\begin{document}
	\maketitle
\section*{Learning objectives}
\begin{itemize}
	\item Knowlegde
	\begin{itemize}
		\item explain relevant data types and their representation for statistical analysis
		\item explain probabilities and random variables
		\item explain distributions of random variables
		\item explain inference and hypothesis testing
		\item explain how data may be collected from experiments involving randomness 
	\end{itemize}
	\item Skills
	\begin{itemize}
		\item choose an appropriate experimental design in respect to a given task
		\item perform statistical analyzes on data collected
		\item use a statistical tool for analysis and visualization of data 
	\end{itemize}
	\item Competence
	\begin{itemize}
		\item use statistical methods and tools to interpret experimental data 
	\end{itemize}
\end{itemize}
\newpage

\section{Lecture 1}
\begin{table}[h]
\centering
\caption{Terms in statistics}
\begin{tabular}{|l|l|} 
	\hline
	Term       & Description                                 \\ 
	\hline
	Variable   & Characteristic or value that can change     \\ 
	\hline
	Data       & The values variables assume                 \\ 
	\hline
	Population & The subjects (human or otherwise) we study  \\ 
	\hline
	Sample     & Subset of the population                    \\
	\hline
\end{tabular}
\end{table}
\begin{itemize}
	\item Descriptive statistics vs Inferential Statistics
		\begin{itemize}
			\item Descriptive statistics: Used to describe data
			\item Inferential statistics: Used to make conclusions about  
		\end{itemize}
\end{itemize}
\paraheading{Measures of central tedency (london)}
\begin{itemize}
	\item Mean
		\begin{itemize}
			\item Division and sum of all values.
			\item Calculated $\overline{X} = \frac{X_1+X_2+...+X_n}{n}=\frac{\sum X}{n}$
			\item Properties of Mean
			\begin{itemize}
				\item Uses all data values
				\item Unique, usually not part of the values
				\item Affected by extremely low or high values (outliers)
			\end{itemize}
		\end{itemize}
	\item Median
	\begin{itemize}
		\item Midpoint of the dataset.
		\item Calculated by sorting all values in ascending order and then selecting the middle one.
		\item If the number of values is odd it will be one value, if the number of values is even it will be the average of two.
		\item Properties of Median:
		\begin{itemize}
			\item Affected less than the mean by extremely low or high values.
		\end{itemize}
	\end{itemize}
	\item Mode
	\begin{itemize}
		\item The Mode is the value that appears most often in the dataset. 
		\item Said to be the most typical case. 
		\item There may be no mode (all unique)m, one mode (unimodal), two modes (bimodal), or many modes (multimodal).
		\item Calculated by sorting all the values, count instances and then select the one (or multiple) that has the most.
		\item Properties of the Mode:
		\begin{itemize}
			\item Easy to compute
			\item Can be used with nominal data
			\item May not exist
		\end{itemize}
	\end{itemize}
	\item Midrange
	\begin{itemize}
		\item The midrange is the average of the lowest and highest value in the dataset.
		\item Calculated by $MR=\frac{Lowest+Highest}{2}$.
		\item Properties of the Midrange:
		\begin{itemize}
			\item Easy to compute
			\item Affected by \textbf{extremely} by low and high values in a dataset.
		\end{itemize}
	\end{itemize}
\end{itemize}

\section{Lecture 2}
\paraheading{Measures of variablity(dispersion)}
\begin{itemize}
	\item Range
	\begin{itemize}
		\item Difference between highest and lowest values in the dataset.
		\item $Highest-Lowest$
	\end{itemize}
	\item Variance
	\begin{itemize}
		\item Together with standard diviation, it is the measure of how spread out your data is.
		\item Variance is the avarage of the squares of distance of each value is from the mean.
		\item Population variance: $\sigma^2 = \frac{\sum (X-\mu)^2}{N}$ where $X$ is the value, $\mu$ is the mean and $N$ is the number of values.
		\item Samlpe variance: $s^2 = \frac{\sum (X - \bar{X})^2}{n-1}$
	\end{itemize}
	\item Standard Diviation
	\begin{itemize}
		\item Together with standard diviation, it is the measure of how spread out your data is.
		\item Population Standard deviation is $\sigma = \sqrt{\frac{\sum (X-\mu)^2}{N}}$
		\item Sample Standard Deviation: $s = \sqrt{\frac{\sum (X - \bar{X})^2}{n-1}}$ where $X$ is the data, $\bar{X}$ is the mean and $n-1$ is the dataset size minus 1.
	\end{itemize}
	\item Coefficient of variation
	\begin{itemize}
		\item the coefficient of variation is the standard deviation divided by the mean expressed as percentage
		\item $CV = \frac{s}{\bar{X}}\cdot 100\%$
	\end{itemize}
\end{itemize}
\paraheading{Measure of position}
Measures of position indicate the position of a value relative to other values in a set of observations
\begin{itemize}
	\item Z-score
	\begin{itemize}
		\item Z score determines how many standard deviations a value is from the mean
		\item $z = \frac{x_i - \bar{x}}{s}$ where $x_i$ is the value, $\bar{x}$ is the mean and $s$ is the standard deviation.
	\end{itemize} 
	\item Percentile
	\begin{itemize}
		\item Percentiles separate the data set into 100 equal groups
		\item A percentile rank for a datum represents the percentage of data values below the datum
	\end{itemize}
	\item Decile and Quartile
		\begin{itemize}
			\item Deciles - seperate the data set into 10 equal groups
			\item Quartiles - seperate the data into 4 equal groups
			\begin{itemize}
				\item $Q_1 = p_{25}$, $Q_2=MD$, $Q_3 = P_{75}$
				\item $Q_2 = median(Low, High)$, $Q_1 = median(Low, Q_2)$, $Q_3 = median(Q_2, High)$ 
				\item The Interquartile Range $IQR = Q_3 - Q_1$
			\end{itemize}
		\end{itemize}
	\item Outlier
	\begin{itemize}
		\item Outlier is an extremely low and high data values when compared to other values
		\item Following data values can be considered outliers:
		\begin{itemize}
			\item less than $Q_1 - 1.5(IQR)$
			\item greater than $Q_3 + 1.5(IQR)$
		\end{itemize}
	\end{itemize}
\end{itemize}

\end{document}





