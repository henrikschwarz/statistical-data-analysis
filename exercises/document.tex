\documentclass{article}

\author{Henrik Schwarz}

\usepackage{amssymb,amsmath}
\usepackage{graphicx}

\graphicspath{ {./media/} }

\title{Statistical Data Analysis \\ \small{EKA: T510028102}}

\date{}

\newcommand{\paraheading}[1]{\paragraph{#1} \mbox{}} % create a paragraph and then a mbox so breaking line is possible
\newcommand{\subparaheading}[1]{\subparagraph{#1} \mbox{}}

\begin{document}
	\maketitle
	\section*{Learning objectives}
	\begin{itemize}
		\item Knowlegde
		\begin{itemize}
			\item explain relevant data types and their representation for statistical analysis
			\item explain probabilities and random variables
			\item explain distributions of random variables
			\item explain inference and hypothesis testing
			\item explain how data may be collected from experiments involving randomness 
		\end{itemize}
		\item Skills
		\begin{itemize}
			\item choose an appropriate experimental design in respect to a given task
			\item perform statistical analyzes on data collected
			\item use a statistical tool for analysis and visualization of data 
		\end{itemize}
		\item Competence
		\begin{itemize}
			\item use statistical methods and tools to interpret experimental data 
		\end{itemize}
	\end{itemize}
	\newpage
	\section{Exercises lecture 1}
	\paraheading{1)}
	The table below shows the height of students in class A (total of 15 students) and class B (total of 16 students), measured in centimeters. For each of the classroom, calculate the following:
	\begin{enumerate}
		\item Median
		\item Mean
		\item Mode
		\item Midrange
	\end{enumerate}
	\begin{table}[h]
		\centering
		\begin{tabular}{|l|l|} \hline
			Class a & Class b \\ \hline
			156 & 185 \\ \hline
			175 & 175 \\ \hline
			189 & 169 \\ \hline
			165 & 182 \\ \hline
			160 & 179 \\ \hline
			154 & 163 \\ \hline
			158 & 191 \\ \hline
			170 & 182 \\ \hline
			171 & 180 \\ \hline
			169 & 174 \\ \hline
			180 & 161 \\ \hline
			175 & 180 \\ \hline
			172 & 176 \\ \hline
			169 & 174 \\ \hline
			162 & 182 \\ \hline
			& 173 \\ \hline
		\end{tabular}
	\end{table}
	\paraheading{Answer}
	
	\subparaheading{Median}
	To get the median we sort the data in ascending order and then selecting the middle one:
	\begin{align*}
		Sort(class_a) = [154, 156, 158, 160, 162, 165, 169, 169, 170, 171, 172, 175, 175, 180, 189]\\
		\text{Middle value} = 169
	\end{align*}
	Since class b has 16 students we get the median by selecting the two middle value and then dividing them by 2:
	\begin{align*}
		&Sort(class_b) = [161, 163, 169, 173, 174, 174, 175, 176, 179, 180, 180, 182, 182, 182, 185, 191] \\
		&\text{Middle values} = [179, 180] \\
		&Median = \frac{179+180}{2}=179.5
	\end{align*}
	
	\subparaheading{mean} \\
	To get the mean we first sum all the values and then divide by:
	
	\begin{align*}
		Class_a=[x_1,x_2,..x_{15}] \\
		\sum Class_a = \sum x_1,...,x_{15} = 2525 \\↓
		\frac{2525}{15} = 168.33
	\end{align*}
	We now do the same for class b:
	\begin{align*}
		\sum class_b = 2826 \\
		\frac{2826}{16} = 176.625
	\end{align*}
	
	\subparaheading{Mode}
	To calculate the mode we count the instances of data values and pick the most common ( most instances ):
	\begin{align*}
		\text{count instances class a} = [(169, 2), (175, 2)]
	\end{align*}
	So class a is bimodal.
	
	Now we caluclate class b:
	\begin{align*}
		\text{count instances class a} = (182,3)
	\end{align*}
	
	\subparaheading{Midrange}
	To calculate the midrange we add the lowest and the highest value togther and divide by two:
	\begin{align}
	Midrange(class_a) = \frac{154+180}{2} = 171.5 \\
	Midrange(class_b) = \frac{161+191}{2} = 167
	\end{align}
\end{document}




















